\vspace{40px} \section{Channel coding and decoding}
The Hamming encoding and decoding is the most crucial part of the transmission because it is the one responsible for the error correction of the transmission. There are two types of hamming encoding, group coding and cyclic coding: in this case, cyclic coding has been utilized to perform the error correction. It is important to note that the channel coding adds some bits to the codeword: precisely during the encoding phase to the codeword are added 5 symbols, making the sequence length a perfect divisor of $26 + 5 = 31$, meanwhile after the decoding the five symbols at the end of the codeword are removed making it again a perfect divisor of 26. It is important to highlight that the Hamming coding (both the group and cyclic) is able to correct only one error per codeword. If there is more than one error, the code will not only not correct the error but it will create other errors by attempting the correction, as already explained on section \ref{two-errors-correction}. This is one of the reasons the \hyperref[interleaving-process]{interleaving} process is needed.





\subsection{Cyclic Hamming encoding}\label{hamming-encoding}
To perform the Hamming encoding to the sequence it is necessary to generate the \texttt{encoding\_matrix} using the \texttt{cyclgen} function\footnote{Note that the function needs to be imported: \texttt{import communications.*}.} that uses the codeword length and the generation polynomial defined at the \hyperref[initial-parameters]{beginning of the code}. The \texttt{hamming\_encoding} function, uses the generated encoding matrix to perform a matrix multiplication and encode the codeword.

\begin{lstlisting}
function encoded_data_matrix = hamming_encoding(binary_data_matrix, codeword_length, k, generation_polynomial)
    % Calculate the number of redundant symbols (parity symbols)
    r = codeword_length - k;
    
    % Generate the cyclic encoding matrix based on the generator polynomial
    [~, cyclic_encoding_matrix] = cyclgen(codeword_length, generation_polynomial);
    
    % Reorder the encoding matrix to match Hamming code requirements
    reorder = [r + 1 : codeword_length, 1 : r];
    cyclic_encoding_matrix = cyclic_encoding_matrix(:, reorder);
    
    % Calculate control symbol values using matrix multiplication
    % Use rem() function to find modulo 2 sum as a remainder of division by 2
    encoded_data_matrix = rem(binary_data_matrix * cyclic_encoding_matrix, 2);
end
\end{lstlisting}


\subsection{Cyclic Hamming decoding}\label{hamming-decoding}
The decoding algorithm is slightly more complicated due to its error-correction properties. The below-displayed \texttt{hamming\_decoding} function has the purpose of analyzing the encoded data and, by calculating the \texttt{syndrome\_value}, performing the error correction. After the decoding, the codeword length won't be 31 anymore but will return to 26.

\begin{lstlisting}
function decoded_data_matrix = hamming_decoding(encoded_data_matrix, codeword_length, k, generation_polynomial)  
    % Determine the number of control symbols
    r = codeword_length - k;
    
    % Specify syndrome calculation matrix
    [~, cyclic_encoding_matrix] = cyclgen(codeword_length, generation_polynomial);
    syndrome_matrix = cyclic_encoding_matrix(:, (1:r));
    syndrome_matrix = [syndrome_matrix; eye(r)];
    
    % Calculate syndrome for each codeword
    syndrome_value = rem(encoded_data_matrix * syndrome_matrix, 2);
    syndrome_value = syndrome_value * 2.^(r - 1 : -1 : 0)';
    
    % Calculate error indexes based on syndrome values
    error_indexes = get_error_indexes(syndrome_matrix, syndrome_value, codeword_length);
        
    % Define error vector table
    error_vector = [zeros(1, codeword_length);
                    eye(codeword_length)];
    
    % Perform correction by adding the error vector to the received codeword
    codeword = rem(encoded_data_matrix + error_vector(error_indexes + 1, :), 2);
    
    % Read the columns of data symbols to form decoded data
    decoded_data_matrix = codeword(:, 1:k);
end
\end{lstlisting}

\noindent Noticeably, a crucial part of the cyclic error correction algorithm is the calculation of the \texttt{error\_indexes}. In the cyclic coding, the syndrome value and the error position are not linearly associated. For example, the syndrome value \texttt{0 0 0 0 1}$_2$ = 1 does not mean that the error is at index 1, but it is in position 31 instead and the syndrome \texttt{1 1 1 1 1}$_2$ = 31 is not associated with the position 31 but with the index 16. The association between the syndrome value and the error position is deterministic and the \texttt{get\_error\_indexes} function helps to associate the error-index and the syndome decimal value.

\begin{lstlisting}
function error_indexes = get_error_indexes(syndrome_matrix, syndrome_value, codeword_length)
    % Initialize vector to store decimal values of binary syndrome patterns
    syndrome_decimal_value_vector = [];

    % Convert binary syndrome values to decimal for association
    for i = 1 : codeword_length
        syndrome_decimal_value_vector = [syndrome_decimal_value_vector; bin2dec(num2str(syndrome_matrix(i, :)))];
    end

    % Combine positions and corresponding decimal values for sorting
    associations = [transpose(1:codeword_length), syndrome_decimal_value_vector];

    % Sort associations based on decimal values for efficient error detection
    associations = sortrows(associations, 2);

    % Create correction index for mapping syndrome values to error positions
    correction_index = [0, associations(:, 1)'];

    % Map syndrome values to error positions using correction index
    error_indexes = correction_index(syndrome_value + 1);
end
\end{lstlisting}
