\vspace{40px} \section{Channel coding and decoding}
The Hamming encoding and decoding is the most crucial part of the transmission because it is the one responsible for the error correction of the transmission. There are two types of hamming encoding, group coding and cyclic coding: in this case, cyclic coding has been utilized to perform the error correction. It is important to note that the channel coding adds some bits to the codeword: precisely during the encoding phase to the codeword are added 5 symbols, making the sequence length a perfect divisor of $26 + 5 = 31$, meanwhile after the decoding the five symbols at the end of the codeword are removed making it again a perfect divisor of 26. It is important to highlight that the Hamming coding (both the group and cyclic) is able to correct only one error per codeword. If there is more than one error, the code will not only not correct the error but will create other errors by attempting the correction, as already explained in section\,\ref{two-errors-correction}. This is one of the reasons the \hyperref[interleaving-process]{interleaving} process is needed.





\subsection{Cyclic Hamming encoding}\label{hamming-encoding}
To perform the Hamming encoding to the sequence it is necessary to generate the \texttt{encoding\_matrix} using the \texttt{cyclgen} function\footnote{Note that the function needs to be imported: \texttt{import communications.*}.} that uses the codeword length and the generation polynomial defined at the \hyperref[initial-parameters]{beginning of the code}. The \texttt{hamming\_encoding} function, uses the generated encoding matrix to perform a matrix multiplication and encode the codeword. Noticeably, the function needs as input and returns as output a sequence even though inside it the sequence is transformed into a matrix and then converted into a vector.

\begin{lstlisting}
function encoded_data = hamming_encoding(binary_data, codeword_length, k, generation_polynomial)
    % Reshape the input binary data into a matrix with k columns
    binary_data_matrix = reshape(binary_data, k, length(binary_data) / k)';
    
    % Number of redundancy symbols (parity bits)
    r = codeword_length - k;

    % Generate the cyclic encoding matrix based on the generator polynomial
    [~, cyclic_encoding_matrix] = cyclgen(codeword_length, generation_polynomial);
    
    % Reorder the encoding matrix to match Hamming code requirements
    reorder = [r + 1 : codeword_length, 1 : r];
    cyclic_encoding_matrix = cyclic_encoding_matrix(:, reorder);
    
    % Calculate control symbol values using matrix multiplication
    % Perform modulo 2 operation to ensure binary result
    encoded_data_matrix = rem(binary_data_matrix * cyclic_encoding_matrix, 2)';
    
    % Reshape the encoded data matrix into a vector and output
    encoded_data = encoded_data_matrix(:)';
end
\end{lstlisting}


\subsection{Cyclic Hamming decoding}\label{hamming-decoding}
The decoding algorithm is slightly more complicated due to its error-correction properties. The below-displayed \texttt{hamming\_decoding} function has the purpose of analyzing the encoded data and, by calculating the \texttt{syndrome\_value}, performing the error correction. After the decoding, the codeword length won't be 31 anymore but will return to 26.

\begin{lstlisting}
function decoded_data = hamming_decoding(encoded_data, codeword_length, k, generation_polynomial)
    % Reshape the encoded data into a matrix where each row is a codeword
    encoded_data_matrix = reshape(encoded_data, codeword_length, length(encoded_data) / codeword_length)';

    % Calculate the number of control symbols (parity bits)
    r = codeword_length - k;
    
    % Generate the syndrome calculation matrix
    [~, cyclic_encoding_matrix] = cyclgen(codeword_length, generation_polynomial);
    syndrome_matrix = cyclic_encoding_matrix(:, (1:r));
    syndrome_matrix = [syndrome_matrix; eye(r)];
    
    % Calculate the syndrome for each codeword
    syndrome_value = rem(encoded_data_matrix * syndrome_matrix, 2);
    syndrome_value = syndrome_value * 2.^(r - 1 : -1 : 0)';
    
    % Get the associations vector (syndrome values to error positions mapping)
    associations = get_associations(syndrome_matrix, codeword_length);
    
    % Map the syndrome value to error position
    correction_index = [0, associations(:, 1)'];
    error_indexes = correction_index(syndrome_value + 1);
    
    % Define the error vector table
    error_vector = [zeros(1, codeword_length);
                    eye(codeword_length)];
    
    % Correct the errors in the received codewords
    codeword = rem(encoded_data_matrix + error_vector(error_indexes + 1, :), 2);
    
    % Extract the information symbols from the corrected codewords
    decoded_data_matrix = codeword(:, 1:k)';
    
    % Reshape the decoded data matrix back to a vector
    decoded_data = decoded_data_matrix(:)';
end
\end{lstlisting}

\noindent Noticeably, a crucial part of the cyclic error correction algorithm is the calculation of the \texttt{associations}. In the cyclic coding, the syndrome value and the error position are not linearly associated. For example, the syndrome value \texttt{0 0 0 0 1}$_2$ = 1 does not mean that the error is at index 1, but it is in position 31 instead and the syndrome \texttt{1 1 1 1 1}$_2$ = 31 is not associated with the position 31 but with the index 16. The association between the syndrome value and the error position is deterministic and the \texttt{get\_associations} function helps to associate the error-index and the syndome decimal value. We can observe that the vector is \texttt{persistent}, meaning that it needs to be calculated only one time, helping the script to be more efficient.

\begin{lstlisting}
function associations = get_associations(syndrome_matrix, codeword_length)
    % Persistent variable to store the associations across function calls
    persistent cached_associations;
    
    % Check if associations are already calculated
    if isempty(cached_associations)
        % Initialize positions and syndrome decimal value vector
        positions = (1 : codeword_length)';
        syndrome_decimal_value_vector = [];
        
        % Calculate syndrome decimal values for each position
        for i = 1 : codeword_length
            syndrome_decimal_value_vector = [syndrome_decimal_value_vector; bin2dec(num2str(syndrome_matrix(i, :)))];
        end
        
        % Create the associations matrix and sort by syndrome value
        cached_associations = [positions, syndrome_decimal_value_vector];
        cached_associations = sortrows(cached_associations, 2);
    end
    
    % Return the cached associations
    associations = cached_associations;
end
\end{lstlisting}
