\vspace{40px} \section{Source coding and decoding}
The second step is to implement an algorithm that will encode and decode the newly generated symbols using the Shannon-Fano algorithm. The two algorithms are based on the results obtained and displayed in the \texttt{Code} column of the \hyperref[tab:shannon-fano]{table} in the \hyperref[source-encoding]{"Source encoding" section}.

\subsection{Shannon-Fano encoding}
The Shannon-Fano encoding can be achieved by using a very simple switch. Sure enough, the helper function \texttt{encode\_symbol} displayed below associates with every symbol in the alphabet and its respective codeword.

\begin{lstlisting}
function result = encode_symbol(symbol)
    % Use a switch statement to assign the encoded representation based on the input symbol
    switch symbol
        case 1
            result = [ 1 0 0 ];
        case 2
            result = [ 0 1 0 0 ];
        case 3 
            result = [ 0 1 0 1 ];
        case 4 
            result = [ 0 0 0 0 0 ];
        case 5 
            result = [ 0 0 1 1 ];
        case 6 
            result = [ 0 0 1 0 ];
        case 7 
            result = [ 1 1 0 ];
        case 8 
            result = [ 1 1 1 ];
        case 9 
            result = [ 1 0 1 ];
        case 10 
            result = [ 0 0 0 1 ];
        case 11 
            result = [ 0 1 1 ];
        case 12
            result = [ 0 0 0 0 1 ];
    end
end
\end{lstlisting}

\noindent The \texttt{shannon\_fano\_encoding} function takes the \texttt{symbol\_sequence} as input and encodes it symbol-by-symbol using the aforementioned \texttt{encode\_symbol} function.

\begin{lstlisting}
function encoded_sequence = shannon_fano_encoding(symbol_sequence)
    encoded_sequence = [];
    
    % Iterate through the symbol sequence and encode each symbol
    for i = 1:length(symbol_sequence)
        encoded_sequence = [encoded_sequence, encode_symbol(symbol_sequence(i))];
    end
end
\end{lstlisting}

\subsection{Shannon-Fano decoding}
The decoding algorithm performs the exact reverse operation of the encoding algorithm. Sure enough, there is the \texttt{decode\_symbol} function which translates the binary sequence into its respective symbol.
\begin{lstlisting}
function symbol = decode_symbol(code)
    % Use a switch statement to check each possible code and return the corresponding symbol
    switch code
        case '[1 0 0]'
            symbol = 1;
        case '[0 1 0 0]'
            symbol = 2;
        case '[0 1 0 1]'
            symbol = 3;
        case '[0 0 0 0 0]'
            symbol = 4;
        case '[0 0 1 1]'
            symbol = 5;
        case '[0 0 1 0]'
            symbol = 6;
        case '[1 1 0]'
            symbol = 7;
        case '[1 1 1]'
            symbol = 8;
        case '[1 0 1]'
            symbol = 9;
        case '[0 0 0 1]'
            symbol = 10;
        case '[0 1 1]'
            symbol = 11;
        case '[0 0 0 0 1]'
            symbol = 12;
        otherwise
            symbol = []; % Return empty if the code does not match any known code
    end
end
\end{lstlisting}

\noindent This function is used in the \texttt{shannon\_fano\_decoding} function wich performs the decoding of the input \texttt{encoded\_sequence}. The body of the function is a little more complicated than the encoding function because the length of the encoded symbol is not fixed - it can be 3, 4 or 5 - and, as such, every time a new symbol is read from the decoded data, a check should be done to understand if the symbol can be decoded or not. 

\begin{lstlisting}
function decoded_sequence = shannon_fano_decoding(encoded_sequence)
    decoded_sequence = [];
    
    % Iterate through the encoded sequence and decode each symbol
    current_code = [];
    for i = 1:length(encoded_sequence)
        current_code = [current_code, encoded_sequence(i)];
        
        % Check if the current code matches any known code
        symbol = decode_symbol(string(mat2str(current_code)));
        
        % If a symbol is found, add it to the decoded sequence and reset the current code
        if ~isempty(symbol)
            decoded_sequence = [decoded_sequence, symbol];
            current_code = [];
        end
    end
end
\end{lstlisting}
