\vspace{40px} \section{Interleaving and deinterleaving}
The interleaving and deinterleaving process is needed to prevent group errors, also called \textsl{burst errors}. This is achieved by deterministically mixing the sequence before the transmission and recomposing it by performing the initial algorithm in reverse. In such a way, if during the communication a burst error happens when the sequence is restored in the initial order, the possibility of having these types of errors drastically decreases.


\subsection{Interleaving}
The \texttt{interleaving} function uses a matrix transpose algorithm to mix the sequence. Noticeably the unmixed sequence is compressed into the \texttt{interleaver\_matrix}: the first 31 symbols of the sequence are inserted into the first column, the second 31 symbols are inserted into the second column and so on. After filling the matrix, to create the interleaved sequence it is necessary to read the matrix through the rows: this is the purpose of the second \texttt{for} loop.

\begin{lstlisting}
function mixed_sequence = interleaving(unmixed_sequence)
    % Define the length of each column (also the number of rows)
    column_length = 31;

    % Calculate the length of each row (also the number of columns) based on the input sequence length
    row_length = length(unmixed_sequence) / column_length;

    % Write on the columns and read on the rows to create the interleaved matrix
    interleaver_matrix = [];

    % Iterate through the rows
    for i = 1 : row_length
        % Extract the current column from the unmixed sequence
        current_column = unmixed_sequence(column_length * (i-1) + 1 : column_length * i)';
        
        % Append the current column to the matrix
        interleaver_matrix = [interleaver_matrix, current_column];
    end
    
    % Initialize the interleaved sequence
    mixed_sequence = [];

    % Iterate through the columns of the matrix
    for i = 1 : column_length
        % Append the elements from each row of the current column to the interleaved sequence
        mixed_sequence = [mixed_sequence, interleaver_matrix(i, 1 : end)];
    end

end
\end{lstlisting}

\subsection{Deinterleaving}
The \texttt{deinterleaving} function is the exact opposite of the interleaving function. In this case, the sequence is inserted into the rows of the \texttt{deinterleaver\_matrix} and then read through the column. Reasonably the body of the function is very similar to its reverse function.

\begin{lstlisting}
function unmixed_sequence = deinterleaving(mixed_sequence)
    % Define the length of each column (also the number of rows)
    column_length = 31;

    % Calculate the number of columns (also the number of rows) based on the input sequence length
    row_length = length(mixed_sequence) / column_length;

    % Initialize the matrix for deinterleaving
    deinterleaver_matrix = [];

    % Iterate through the columns of the interleaved sequence
    for i = 1 : column_length
        % Extract the current column from the mixed sequence
        current_column = mixed_sequence(row_length * (i-1) + 1 : row_length * i)';
        
        % Append the current column to the matrix
        deinterleaver_matrix = [deinterleaver_matrix, current_column];
    end
    
    % Initialize the deinterleaved sequence
    unmixed_sequence = [];

    % Iterate through the rows of the matrix
    for i = 1 : row_length
        % Append the elements from each column of the current row to the deinterleaved sequence
        unmixed_sequence = [unmixed_sequence, deinterleaver_matrix(i, 1 : end)];
    end

end
\end{lstlisting}


