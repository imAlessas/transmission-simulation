\vspace{40px} \section{Interleaving and deinterleaving}
The interleaving and deinterleaving process is needed to prevent group errors, also called \textsl{burst errors}. This is achieved by deterministically mixing the sequence before the transmission and recomposing it by performing the initial algorithm in reverse. In such a way, if during the communication a burst error happens when the sequence is restored in the initial order, the possibility of having these types of errors drastically decreases.


\subsection{Interleaving}


\begin{lstlisting}
function mixed_sequence = interleaving(unmixed_sequence)
    % Define the length of each column (also the number of rows)
    column_length = 31;

    % Calculate the number of length of each row (also the number of columns) based on the input sequence length
    row_length = length(unmixed_sequence) / column_length;

    % Write on the columns and read on the rows to create the interleaved matrix
    interleaver_matrix = [];

    % Iterate through the rows
    for i = 1 : row_length
        % Extract the current column from the unmixed sequence
        current_column = unmixed_sequence(column_length * (i-1) + 1 : column_length * i)';
        
        % Append the current column to the matrix
        interleaver_matrix = [interleaver_matrix, current_column];
    end
    
    % Initialize the interleaved sequence
    mixed_sequence = [];

    % Iterate through the columns of the matrix
    for i = 1 : column_length
        % Append the elements from each row of the current column to the interleaved sequence
        mixed_sequence = [mixed_sequence, interleaver_matrix(i, 1 : end)];
    end

end
\end{lstlisting}

\subsection{Deinterleaving}
