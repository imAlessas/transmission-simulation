\vspace{40px} \section{Padding bits}
One important thing to notice is that the cyclic Hamming encoding is that the algorithm requires an input matrix whose number of elements is a divisor of the information symbols $k$. The problem is that it is not granted that the generated and encoded sequence is a perfect divisor of $k$, so it is crucial to add some padding bits to round the length of the sequence to the closest bigger multiplier of $k$. The idea behind this process is to count how many padding bits are needed to round the length of the sequence and store the value into a $r$-bit sequence containing the binary number of bits to remove during the reception phase.


\subsection{Add padding bits}
To add the padding bits, the first thing to know is how many padding bits are needed to round the length of the sequence. The purpose of the following helper function is to calculate the number of bits to add. The information needed to calculate this value is input parameters that are calculated at the very beginning of the script (see\,\ref{initial-parameters}). In fact, with a codeword length of 31, the number of bits to manage this number is $r = 5$, meaning that the usable information bits are $k = 26$.

\begin{lstlisting}
function number_of_padding_bits = count_padding_bits(compact_sequence, k, r)  
    % Calculate the number of padding bits required
    number_of_padding_bits = k - rem(length(compact_sequence), k);

    % Adjust the number of padding bits if it is less than the storage_bits
    if number_of_padding_bits < r
        number_of_padding_bits = number_of_padding_bits + k;
    end
end
\end{lstlisting}

\noindent After obtaining the number of bits to add, the only thing to do is to fill the sequence and at the end incorporate the binary sequence containing the number of added bits as it is shown in the \texttt{add\_padding\_bits} function below.

\begin{lstlisting}
function padded_sequence = add_padding_bits(compact_sequence, k, r)
    % Determine the number of bits needed for padding
    bits_to_pad = count_padding_bits(compact_sequence, k, r);

    % Convert the number of bits to pad to binary representation
    binary_padding_value = str2num(dec2bin(bits_to_pad, r)')';

    % Add padding bits to the end of the sequence
    padded_sequence = [compact_sequence, zeros(1, bits_to_pad - r), binary_padding_value];
end
\end{lstlisting}


\subsection{Remove padding bits}
On the other hand during the reception, the only thing to do is to get the last $r$ symbols of the sequence and convert them into a decimal number, as shown in the \texttt{get\_padding\_bits} function below.

\begin{lstlisting}
function padding_bits = get_padding_bits(padded_sequence, r)
    % Extract the last r bits from the padded sequence
    bits = padded_sequence(end - r + 1 : end);
    
    % Convert the binary representation of the bits to a string
    str = '';
    for i = 1 : length(bits)
        str = append(str, num2str(bits(i)));
    end
    
    % Convert the binary string to decimal to obtain the number of padding bits
    padding_bits = bin2dec(str);
end
\end{lstlisting}

\noindent After getting the number $n$ of padding bits added, the last step is to remove the last $n$ symbols of the sequence to get the original data.

\begin{lstlisting}
function compact_sequence = remove_padding_bits(padded_sequence, r)
    % Call the get_padding_bits function to determine the number of padding bits
    padding = get_padding_bits(padded_sequence, r);

    % Remove the padding bits from the end of the sequence
    compact_sequence = padded_sequence(1 : end - padding);
end
\end{lstlisting}
