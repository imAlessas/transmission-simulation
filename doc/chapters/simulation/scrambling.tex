\vspace{40px} \section{Scrambling and descrambling}
The scrambling process helps with the synchronization between the sender machine and the receiver device. More precisely, when there is a long sequence of \texttt{"0"} or \texttt{"1"} symbols it becomes rather difficult to understand how many of them are being transmitted. For this purpose, before modulating and transmitting the signal, is extremely helpful to perform a logical \texttt{XOR} to every codeword with a pseudo-random sequence, called \texttt{scambling\_key}. This is the purpose of the \texttt{scrambling} function: it applies an exclusive \texttt{OR} bitwise with a key that is known both from the source and the destination.

\begin{lstlisting}
function scrambled_sequence = scrambling(unscrambled_sequence, scrambler_key)
    % Initialize the output sequence
    scrambled_sequence = [];

    % Determine the length of the scrambler key
    codeword_length = length(scrambler_key);

    % Loop through the input sequence in codeword-sized chunks
    for i = 1 : length(unscrambled_sequence) / codeword_length
        % Extract the current codeword from the input sequence
        current_codeword = unscrambled_sequence(codeword_length * (i - 1) + 1 : codeword_length * i);
        
        % Perform XOR operation with the scrambler key
        scrambled_codeword = xor(current_codeword, scrambler_key);

        % Append the scrambled codeword to the output sequence
        scrambled_sequence = [scrambled_sequence, scrambled_codeword];
    end
end
\end{lstlisting}

\noindent There is no need to create the \texttt{descrambling} function because the \texttt{XOR} operation is cyclical: performing this operation two times to a sequence of binary numbers will return the initial sequence. In this case, the function has been created just for better clarity and readability.

\begin{lstlisting}
function unscrambled_sequence = descrambling(scrambled_sequence, scrambler_key)
    % Utilize the scrambling function in reverse to perform descrambling
    unscrambled_sequence = scrambling(scrambled_sequence, scrambler_key);
end
\end{lstlisting}

