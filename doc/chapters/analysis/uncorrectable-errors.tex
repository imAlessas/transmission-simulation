\vspace{40px} \section{Uncorractable errors} \label{uncorrectable}
The reason for calculating the probability of 2 or more errors occurring in the same codeword has been explained in the previous chapters. The main reason is that the Hamming code is not able to correct 2 or more errors in the codeword: this does not mean that at least one of them is correct but it leads to the creation of another error. Consequently, the probability of an uncorrectable error is strictly bonded to the fact that with that type of error a new error will be almost surely generated by the Hamming code. For this reason, this probability should be as near to zero as possible.

To calculate such a value the mathematization equation below-displayed should be computed:

\begin{equation*}
    P_{\geq2\,err} = 1 - (1 - P_{err})^m - \sum_{i = 1}^g\, C^i_m\,P_{err}^i(1 - P_{err})^{m - i}
\end{equation*}

\noindent In this particular case, $g = 1$ and $C^i_m = m$ so the equation can be simplified as follows:

\begin{equation*}
    P_{\geq2\,err} = 1 - (1 - P_{err})^m - m\,P_{err}^i(1 - P_{err})^{m - 1}
\end{equation*}

\noindent Which translates in the following MATLAB line:

\begin{lstlisting}
    % probability of the case when it is not possible to correct errors with the Hamming code (>= 2 errors)
    P_uncor = 1 - (P_err_comp)^(codeword_length) - codeword_length * P_err * (P_err_comp)^(codeword_length - 1);
\end{lstlisting}

\noindent After running the script the probability of an uncorrectable error $P_{\geq2\,err} = 1.2338\cdot10^{-5}$, which is a low value but, with a high mole of transmitted data there is the possibility to still occur in uncorrectable errors that may lead to an unsuccessful transmission. The probability is still rather low but it is a still possible scenario that should be taken into consideration.

