\vspace{40px} \section{Error correction} \label{ciclic-coding}
The error correction analysis is important to understand how powerful and yet dangerous the error correction codes are. In this document, the analysis focuses on the \textsl{cyclic Hamming code} error correction properties even though the conclusions are still valid for the \textsl{group Hamming code} (both systematic and non-systematic). 

Before analyzing the error correction code, it is necessary to properly implement it. The first thing to do is to generate a binary sequence and the encoding and decoding matrix. To do so it has been used the \texttt{cyclgen} function which should be imported from the communication package as follows: \texttt{import communications.*}. These steps are summarized in the code snipped below.

\begin{lstlisting} 
    % generate a sequence of k binary symbols
    binary_sequence =  randi(2, 1, k) - 1;

    % generate decoding and encoding matrix
    [cyclic_decoding_matrix, cyclic_encoding_matrix] = cyclgen(codeword_length, generation_polynomial);

    % reorder the matrixes
    % [6 -> 31, 1 ->]
    reorder = [6:codeword_length, 1:5];

    cyclic_encoding_matrix = cyclic_encoding_matrix (:, reorder); 

    cyclic_decoding_matrix = (cyclic_decoding_matrix (:, reorder))'; 
\end{lstlisting}

\noindent One crucial thing to do is to redefine the associations between the syndrome values and the error position. The vector that is shown below is not random at all but it has been calculated using the algorithm shown in the chapter\,\ref{hamming-decoding} and the copy-pasted it. This vector is very important because if it is not defined the correction algorithm won't work at all but will increase the error rate.

\begin{lstlisting} 
    % Associates the syndrome to the bit.
    % This vector has been calculated in the hamming_decoding function and copy-pasted here.
    associations = [0 31 30 13 29 26 12 20 28 2 25 4 11 23 19 8 27 21 1 14 24 9 3 5 10 6 22 15 18 17 7 16];
\end{lstlisting} 

\noindent After setting up the error correction algorithm, it is possible to begin the analysis by encoding the codeword and studying the behavior of the cyclic Hamming code. Reasonably, by introducing no errors the decoded codeword is the same as the initial codeword.

\begin{lstlisting} 
    % encode the codeword
    codeword = mod(binary_sequence * cyclic_encoding_matrix, 2);
    initial_codeword = codeword;
    
    % decode without errors
    syndrome_no_error = mod(codeword * cyclic_decoding_matrix, 2);
\end{lstlisting}

\noindent For the next analysis, to properly understand the functioning of the error correction cyclic code, it will be used the following 26-symbols randomly-generated binary sequence:

\begin{center}
    \texttt{1 0 1 0 0 1 0 1 0 1 1 0 1 0 1 0 0 1 1 1 1 0 0 1 1 0}
\end{center}

\noindent The above sequence, after the cyclic hamming encoding will have 31 symbols as follows:
\begin{center}
    \texttt{1 0 1 0 0 1 0 1 0 1 1 0 1 0 1 0 0 1 1 1 1 0 0 1 1 0 }\textcolor{blue}{\texttt{0 0 1 0 0}}
\end{center}

\subsection{One error}
By introducing one error to a random position it is necessary to calculate the decimal syndrome value and then use the \texttt{associations} vector to detect and correct the error. The code snippet presented below sums up the error correction after one error is displayed below.

\begin{lstlisting}
    % introduce one error
    error_position = randi(31, 1);
    codeword(error_position) =~ codeword(error_position);

    codeword_one_error = codeword;

    % get the error syndrome
    syndrome_one_error = mod(codeword_one_error * cyclic_decoding_matrix, 2);

    % convert the syndrome into decimal
    syndrome_one_error_decimal = bin2dec(num2str(syndrome_one_error));

    % get the index of the wrong symbol
    wrong_symbol_position = associations(syndrome_one_error_decimal + 1);

    % correct the error
    codeword_one_error(wrong_symbol_position) = ~codeword_one_error(wrong_symbol_position);
\end{lstlisting}

\noindent By introducing one error in a random position, like position 30, the wrong sequence would be the following:

\begin{center}
    \texttt{1 0 1 0 0 1 0 1 0 1 1 0 1 0 1 0 0 1 1 1 1 0 0 1 1 0 }\textcolor{blue}{\texttt{0 0 }}\textcolor{red}{\texttt{0 }}\textcolor{blue}{\texttt{0 0}}
\end{center}

\noindent Nonetheless the code manages to spot the error using the syndrome value:

\begin{center}
    \texttt{0 0 0 1 0}
\end{center}

\noindent It must be highlighted again the importance of the \texttt{associations} vector because the decimal value of the syndrome is not 30 but it is 2. The vector bonds the decimal value 2 to the position error 30 successfully managing to perform the error correction:
\begin{center}
    \texttt{1 0 1 0 0 1 0 1 0 1 1 0 1 0 1 0 0 1 1 1 1 0 0 1 1 0 }\textcolor{blue}{\texttt{0 0 1 0 0}}
\end{center}


\subsection{Two errors} \label{two-errors-correction}
By using the below-displayed code, very similar to the previous one, it is possible to introduce a second error to the codeword to analyze the effect of two errors in the codeword. 

\begin{lstlisting}
    % introduce second error
    error_position = randi(31, 1);
    codeword(error_position) =~ codeword(error_position);

    codeword_two_errors = codeword;

    % get the error syndrome
    syndrome_two_errors = mod(codeword_two_errors * cyclic_decoding_matrix, 2);

    % convert the syndrome into decimal
    syndrome_two_errors_decimal = bin2dec(num2str(syndrome_two_errors));

    % get the index of the wrong symbol
    wrong_symbol_position = associations(syndrome_two_errors_decimal + 1);

    % correct the error
    codeword_two_errors(wrong_symbol_position) = ~codeword_two_errors(wrong_symbol_position);
\end{lstlisting}

\noindent By running the code and generating a second error position, like position 11, the codeword becomes the following:

\begin{center}
    \texttt{1 0 1 0 0 1 0 1 0 1 } \textcolor{red}{\texttt{0 }} \texttt{0 1 0 1 0 0 1 1 1 1 0 0 1 1 0 }\textcolor{blue}{\texttt{0 0 }}\textcolor{red}{\texttt{0 }}\textcolor{blue}{\texttt{0 0}}
\end{center}

\noindent Unfortunately in this case the syndrome value will not be helpful:

\begin{center}
    \texttt{0 1 1 1 0}
\end{center}

\noindent which its decimal value is 14, meaning that, using the \texttt{associations} vector, the error position is 19 not corresponding in either the two errors but creating a third error:
\begin{center}
    \texttt{1 0 1 0 0 1 0 1 0 1 } \textcolor{red}{\texttt{0 }} \texttt{0 1 0 1 0 0 1 }\textcolor{red}{\texttt{0 }} \texttt{1 1 0 0 1 1 0 }\textcolor{blue}{\texttt{0 0 }}\textcolor{red}{\texttt{0 }}\textcolor{blue}{\texttt{0 0}}
\end{center}

\noindent For this reason it is important to analyze the probability of two errors occurring in the codeword (see\,\ref{uncorrectable}): because with the error correction code the two errors not only not be correct but also a third error will be generated in the attempt.

\subsection*{Three erros particoular situation}
If three errors occur in specific positions the algorithm may not even detect the errors because the syndrome is zero. This happens when the three error syndromes cancel each other out. In this case, if the errors are at positions 30 and 11, the critical error position is 14. In this situation, the error syndrome is 0, preventing the algorithm from detecting and correcting any of the three errors. The following code calculates the critical position for any two random error positions using a simple \texttt{brute force} algorithm:

\begin{lstlisting}
    % Three errors experiment
    codeword = initial_codeword;

    % introduce error
    error_position = randi(31, 1)
    codeword(error_position) = ~codeword(error_position);

    % introduce error
    error_position = randi(31, 1)
    codeword(error_position) = ~codeword(error_position);

    critical_position = -1;

    for i = 1 : 31
        % introduce error
        codeword(i) = ~codeword(i);

        % memorize the codeword
        codeword_three_errors = codeword;

        % get the error syndrome
        syndrome_three_errors = mod(codeword_three_errors * cyclic_decoding_matrix, 2);
        
        % convert the syndrome into decimal
        syndrome_three_errors_decimal = bin2dec(num2str(syndrome_three_errors));
        
        % get the index of the wrong symbol
        wrong_symbol_position = associations(syndrome_three_errors_decimal + 1);
        
        if ~wrong_symbol_position
            critical_position = i;
        end
    end
\end{lstlisting}

