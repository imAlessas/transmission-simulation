\section{Introduction}
\begin{figure}[t!]
    \centering
    \begin{tikzpicture}[
        Step/.style = {rectangle, rounded corners = 5, inner sep = 10, draw = blue!60, fill = blue!10, thick, blur shadow, minimum size = 1cm},
        % 
        Noise/.style = {rounded rectangle, inner sep = 7, draw = black, dashed, very thick},
        % 
        Message/.style = {rectangle, draw = black!60, fill = black!1, thick, minimum width = 1.7cm, minimum height = 1cm },
    ]
        % Source
        \node[Step] (Source) {\textbf{Source}};
        \node[Step] (FormDevSou) [below = of Source] {Formatting Device};
        \node[Step] (SourceCoding) [below = of FormDevSou] {\hyperref[source-coding]{Source Coding}};
        \node[Step] (ChanCod) [below = of SourceCoding] {\hyperref[hamming-encoding]{Channel Coding}};
        \node[Step] (Interleaving) [below = of ChanCod] {\hyperref[interleaving]{Interleaving}};
        \node[Step] (Scrambling) [below = of Interleaving] {\hyperref[scrambling]{Scrambling}};
        \node[Step] (Modulation) [below = of Scrambling] {\hyperref[modulation]{Modulation}};

        % Noise
        \node[Noise] (Noise) [right = 2cm of Modulation] {Noise};

        % Message
        \node[Message] (Message) [right = 2cm of Source] {Message};

        % Destination
        \node[Step] (Destination) [right = 2cm of Message] {\textbf{Destination}};
        \node[Step] (FormDevDest) [below = of Destination] {Formatting Device};
        \node[Step] (SourceDec) [below = of FormDevDest] {\hyperref[source-decoding]{Source Decoding}};
        \node[Step] (ChanDec) [below = of SourceDec] {\hyperref[hamming-decoding]{Channel Decoding}};
        \node[Step] (Deinterleaving) [below = of ChanDec] {\hyperref[deinterleaving]{Deinterleaving}};
        \node[Step] (Descrambling) [below = of Deinterleaving] {\hyperref[descrambling]{Descrambling}};
        \node[Step] (Demodulation) [below = of Descrambling] {\hyperref[demodulation]{Demodulation}};
        



        % Arrow Message
        \draw[-, dashed] (Source) to node[]{} (Message);
        \draw[->, dashed] (Message) to node[]{} (Destination);


        % Arrows source
        \draw[->, very thick] (Source) to node[]{} (FormDevSou);
        \draw[->, very thick] (FormDevSou) to node[]{} (SourceCoding);
        \draw[->, very thick] (SourceCoding) to node[]{} (ChanCod);
        \draw[->, very thick] (ChanCod) to node[]{} (Interleaving);
        \draw[->, very thick] (Interleaving) to node[]{} (Scrambling);
        \draw[->, very thick] (Scrambling) to node[]{} (Modulation);

        % Noise
        \draw[-, very thick] (Modulation) to node[]{} (Noise);
        \draw[->, very thick] (Noise) to node[]{} (Demodulation);

        % Arrows destination
        \draw[->, very thick] (Demodulation) to node[]{} (Descrambling);
        \draw[->, very thick] (Descrambling) to node[]{} (Deinterleaving);
        \draw[->, very thick] (Deinterleaving) to node[]{} (ChanDec);
        \draw[->, very thick] (ChanDec) to node[]{} (SourceDec);
        \draw[->, very thick] (SourceDec) to node[]{} (FormDevDest);
        \draw[->, very thick] (FormDevDest) to node[]{} (Destination);

    \end{tikzpicture}
    \caption{The diagram of the digital information transmission system.}
    \label{fig:transmission_diagram}
\end{figure}
This document has the goal of illustrating the functioning of a telecommunication system transmission from the beginning to the end. The full schematic - containing every step - of a transmission system is presented in figure \ref{fig:transmission_diagram}. Before exploring the mathematical background hidden between the steps, it is crucial to understand what every phase of the system means.

\begin{itemize}
\renewcommand{\labelitemi}{$\diamond$}
    \item \textsl{Source.} The source device is whichever device is sending a signal; it could be a television, a computer, a smartphone, or anything else.
    % 
    \item \textsl{Formatting Device.} The formatting device's task is to translate the information from analogic to digital which translates into sampling the continuous analogic signal and creating a discrete digital signal that can be transmitted through digital devices.
    % 
    \item \textsl{Source Coding.} 
    % 
    \item \textsl{Channel Coding.} 
    % 
    \item \textsl{Interleaving.} The interleaver is needed to transform package errors into independent errors. This is achieved by changing the ordering of the symbols that will be transmitted.
    % 
    \item \textsl{Scrambling.}
    % 
    \item \textsl{Modulation.} The modulation process' goal is to match the spectrum of the transmitted signal with the transmission channel bandwidth making the signal more noise-immune and increasing the data-transfer rate; these operations are performed by the modulator. There are different types of modulation, the one utilized in this project is the \textsl{Binary Phase Shift Keying}, which is one of the most effective modulations against noise. 
    % 
    \item \textsl{Noise.} The noise is a crucial obstacle to overcome to have a successful transmission; the noise is the main reason for a wrongly transmitted symbol. There are different types of noise, some of them are generated by other transmissions, others are due to the physical medium and others are caused by the intermediate devices between the transmission. Nevertheless, in every transmission, there will be the \textsl{Gaussiam White Noise} which is a thermal noise caused by the Big Bang.
    % 
    \item \textsl{Demodulation.} In this phase the demodulator device, after receiving the disturbed signal, will try to detect the signal to regenerate the original one. Sometimes the noise energy will be stronger than the signal energy generating errors that will be corrected in the next steps.
    % 
    \item \textsl{Descrambling.}
    % 
    \item \textsl{Deinterleaving.} The deinterleaver reorders the transmitted symbols in the opposite way that the interleaver did. In such a way the \textit{burst} errors that occurred during the transmission will become single errors that can be easily recovered.
    % 
    \item \textsl{Channel Decoding.} 
    % 
    \item \textsl{Source Decoding.} 
    % 
    \item \textsl{Formatting Device.} During the transmission this device converts the signal from analogic to digital, during the reception of the signal the formatting device translates the discrete digital signal into a continuous analogic signal. 
    % 
    \item \textsl{Destination.} The destination device is whichever device will receive the signal. Likewise the source one, the destination device could be a satellite, a smartphone, a server, or anything else.
\end{itemize}








